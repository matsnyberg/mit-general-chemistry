\documentclass[../mit-general-chemistry.tex]{subfiles}
\begin{document}


\chapter{Spontaneous change and free energy}



A {\em spontaneous reaction} proceeds in the forward direction without
the need for outside intervention. It is {\em exorgonic}. A reaction
that does not do that is called {\em non-spontaneous} or {\em
  endorgonic}.

An example of a spontaneous reaction is the formation of rust
\ceeqstar{4Fe(s) + 3O2(g) -> 2Fe2O3(s)} with $\enthalpy^0 =
-824~\kjpm$, an exothermic reaction.


The autoionization of water \ceeqstar{H3O+(aq) + OH-(aq) -> 2H2O} is
also a sponatneous reaction. It has $\enthalpy^0 = -55.9~\kjpm$ and is
thus also an exothermic reaction.

Another reaction, extremly important in bio sciences, is the ATP
hydrolysis \ceeqstar{ATP^4-(aq) + 2H2O -> ADP^3-(aq) + HPO4^2-(aq) +
  H3O+} which drives large numbers of reactions in the cells. The
reaction has an $\enthalpy^0 = \SI{-24}{\kilo\joule\per\mol}$.

There are tons of exothermic reactions (with $\enthalpy < 0$) that are
spontaneous and it is easy to draw the conclusion that this is the
criterion for spontaneity, but that is not the case. Reactions can be
spontaneous without being exothermic. The following reactions, for
instance, are endothermic but also sponatneous.

The melting of ice, \ce{H2O (s) -> H2O(l)}, has $\enthalpy^0 =
+6.95~\kjpm$ and is endothermic.

So is \ceeqstar{NH4NO3(s) -> NH4+(aq) + NO3-(aq)} with $\enthalpy^0 =
\SI{+28}{\kilo\joule\per\mol}$.

So we see that enthalpy is not the key to spontaneity. Instead,
spontaneity depends on Gibb's free energy, \gibbs.





\section{Gibb's free energy}



We define Gibb's free energy as
\begin{equation}\label{eq:gibbs}
  \gibbs = \enthalpy - T \Delta S
\end{equation}
where $T$ is the temperature and $\Delta S$ is the change in entropy,
so it is clear why \enthalpy is closely related to \gibbs. The
equation is true for constant temperature and pressure.

The relationship to spontaneity is

\begin{center}
  \begin{tabular}{ll}
    $\gibbs < 0$ & reaction is spontaneous \\
    $\gibbs > 0$ & reaction is not spontaneous \\
    $\gibbs = 0$ & reaction has reached equilibrium  \\
  \end{tabular}
\end{center}


A negative change in energy, it is just like with enthalpy, means that
energy is released as the reaction progresses. That is going to be a
spontaneous reaction.


So how come we need Gibb's free energy and that enthalpy is not enough
to predict spontaneity? If we rewrite Equation~\ref{eq:gibbs} into
\begin{equation*}
  \enthalpy = \gibbs + T \Delta S
\end{equation*}

We see that the energy of a reaction, the enthalpy ($H$), consists of
a term that is the free energy (\gibbs), useful for work, and an
amount of energy that gets ``stuck'' in the system. That energy that
gets ``stuck'' is the $T \Delta S$ term.




\subsection{Entropy}


Entropy, $S$, is an amount of the disorder of a system. We write
$\Delta S$ for the change in entropy. Entropy is a state function.

If $\Delta S > 0$ we are {\em increasing the disorder} of a system
(decreasing the order).

For the phases, or states, of matter the entropy is greater for
gaseous matter than it is for liquid matter and solid matter have the
least entropy. We express this (somewhat loosely) as
\begin{equation*}
  \Delta S_g > \Delta S_l > \Delta S_s
\end{equation*}
for the different phases gaseous, liquid and solid.


Entropy for reactions, $\Delta S_r^0$, can be calculated from the
absolute entropy of the reactants and products in the same way as for
enthalpy.
\begin{equation*}
  \Delta S_r^0 = \sum~S^0_{\text{products}} - \sum~S^0_{\text{reactants}}
\end{equation*}
where $S^0$ is {\em absolute standard entropy}. Since $S = 0$ for the
perfect crystal at $T = \SI{0}{\kelvin}$, we can consider the
existence of the notion of an absolute measure of entropy.



\begin{remark}
  Note that measures of entropy tends to be much smaller than measures
  of enthalpy and other measures of energy. It is therefore common
  that they are given in joules instead of kilo joules and need to be
  converted into kilo joules when they are used in formulas. 
\end{remark}





\subsubsection{Entropy and probability}



Entropy is more than disorder. It is, in fact, a clearly defined,
statistical measure of the arrangement of elements and their
likelihoods of occurrence.

In thermodynamics more likely (probable) states will predominate over
time (higher $S$). This is a consequence of the second law of
thermodynamics that states that the total entropy of an isolated
system never decreases over time. As time goes, entropy will increase
or remain the same.

The second law of thermodynamics has two paragraphs. It states that
entropy of an isolated system never decreases. For reversible
processes it may be constant, but for irreversible processes it will
increase. It also states that the total entropy of any system and of
it's surroundings will always increase.


This does not follow directly from the basic laws of physics, instead
it is a matter of probability and statistics. As there are many more
states with higher levels of entropy than there are with lower levels,
systems tend toward disorder if it changes state independently over
time. Then the changes of state will be a matter of probability where
there are more high-entropy states then there are low-entropy states,
which makes the high-entropy states more likely.


As an example of the second law of thermodynamics we intuitively know
that if we pump in gas in a container, the gas particles will
initially be close to the entry point. Over time the particles will
spread out all over the container. It is highly unlikely that the
particles will cluster back at the entry point again.


There are many more ways (arrangements) for the particles to be spread
out than there are to be ordered.



We are going to consider six coins as a small model. We define states
of the model as the number of heads that
show up. We have the sample space $\mathscr{S} = \{1H, 2H, 3H, 4H, 5H,
6H\}$ for $1H$ meaning a state where all coins but one is showing
``tails'', $2H$ all but two and so on. The number of arrangements in
each state is called the {\em likelihood} of that state and is denoted
by $W$.

We are going to consider the $1H$ (``concentrated'') state and the
$3H$ (``more spread out'') state. We realize that there are six
arrangements in the $1H$ state since there are six coins that can be
the one that is showing ``heads''. We can enumerate the arrangements
of the $1H$ state by simply flipping one coin at a time.

\begin{htable}\label{tbl:coins:states}

  \hspace*{\fill}
  \begin{tabular}{ll}
    HTTTTT & THTTTT \\
    TTHTTT & TTTHTT \\
    TTTTHT & TTTTTH \\
  \end{tabular}
  \hspace*{\fill}

  \vspace{2em}
  
  \hspace*{\fill}
  \begin{tabular}{llll}
    HHHTTT & HHTHTT & HHTTHT & HHTTTH \\
    HTHHTT & HTHTHT & HTHTTH & HTTHHT \\
    HTTHTH & HTTTHH & THHHTT & THHTHT \\
    THHTTH & THTHHT & THTHTH & THTTHH \\
    TTHHHT & TTHHTH & TTHTHH & TTTHHH \\
  \end{tabular}
  \hspace*{\fill}
  
  \caption{ An enumeration of the arrangements of the $1H$ state and
    of the $3H$ state. The conclusion is that there are so much higher
    numbers of arrangements in the $3H$ state than there are in the
    $1H$ state.}
\end{htable}


The $3H$ state, on the other hand, is the state with the most number
of arrangements. There are $\binom{6}{3} = 20$ arrangements where
there are three ``heads'' and three ``tails''.

We can see the arrangements of the two states in
Table~\ref{tbl:coins:states} which illustrates the point that there
are quite a few more arrangements in the $3H$ state than there are in
the $1H$ state. If we would just toss the six coins $3H$ would be the
most common result. In the language of thermodynamics and entropy we
could say that the $3H$ state will dominate over time.



Calling the number of arrangements in each state (6 and 20 in our
example) the {\em likelihood} of that state is a term coined by
Boltzmann.


Boltzmann's great insight was that entropy is related to $W$.
\begin{equation*}
  S = k_B \ln W
\end{equation*}
for Boltzmann's constant $k_B$, the gas constant over Avogadro's number
\begin{equation*}
  k_B = \frac{R_0}{N_A} = 1.38064852(79)\times 10^{−23} \si{\joule\per\kelvin}
\end{equation*}

We can use this formula to to calculate the entropy for the coin
tossing states $1H$ and $3H$.
\begin{align*}
  S_1 &= k_B \ln 6 = 2.47 \times 10^{-23} \\
  S_3 &= k_B \ln 20 = 4.14 \times 10^{-23} \\
  S_3 &> S_1 \\
\end{align*}
Note that the entropy does not depend directly on the total number of
arrangements like probability do.

We can calculate the entropy change ($\Delta S$) out of ``going from
one state to another'' and find that $\Delta S_{3\to 1} = S_3 - S_1 >
0$ and conversely $\Delta S_{1\to 3} = S_1 - S_3 < 0$, that there is
an increase in entropy as we go from he $1H$ state to the $3H$ state,

If we would repeat the coin tossing over an over again, we would find
that the $3H$ outcomes would start to spontaneously overwhelm the $1H$
outcomes.

Going from $1H$ to $3H$ is thus spontaneous as we saw above that
$\Delta S = S_1 - S_3 < 0$, this corresponds to predictions of
thermodynamics.




Let's move to a chemical example, carbonyl sulfide,
\chemfig{O=[,0.7]C=[,0.7]S}. This is a linear molecule with a carbon
atom in the middle bound to one oxygen atom and one sulfur atom in
each direction. At very low temperatures carbonyl sulfide form
crystals where the molecules are lined up, in parallel, next to each
other. In this arrangement, the sulfur atoms point one way or the
other. Let's call sulfur atoms pointing one way ``heads'' and the
other ``tails'' to see the resemblance to our coin tossing example. If
we study one ``line'' of parallel molecules it could be illustrated
such as in the figure below.


\begin{center}
  \begin{tikzpicture}[node distance=.2cm]
    \def\heads{\chemfig{C(=[:90]S)(=[:270]O)}}
    \def\tails{\chemfig{C(=[:90]O)(=[:270]S)}}
    \node (a) {\heads};
    \node[right=of a] (b) {\heads};
    \node[right=of b] (c) {\tails};
    \node[right=of c] (d) {\tails};
    \node[right=of d] (e) {\heads};
    \node[right=of e] (f) {\tails};
    \node[right=of f] (g) {$\cdots$};
    \node[left=of a] {$\cdots$};

    \node[above=1mm of a]{\textbf{H}};
    \node[above=1mm of b]{\textbf{H}};
    \node[above=1mm of c]{\textbf{T}};
    \node[above=1mm of d]{\textbf{T}};
    \node[above=1mm of e]{\textbf{H}};
    \node[above=1mm of f]{\textbf{T}};
  \end{tikzpicture}
\end{center}


In the diagram, the molecules we consider ``heads'' are denoted with
the letter H. The molecules we consider ``tails'' are denoted with the
letter T.

We are now going to consider a mole ($N_A = \num{6.022d23}$) of
molecules instead of six. We will consider the (unique) arrangement of
a perfect crystal with all sulfur atoms ``pointing'' in one direction
and call this heads. This is also a state associated with a single
arrangement, thus $W_{\text{perfect}} = 1$. We will compare this with
a ``random'' one where we don't know the configuration of the
atoms. In this state we have $\binom{N_A}{n}$ arrangements in each
state of $n$ heads. It is obvious that these are {\em huge} numbers.

We can tell that $S_{\text{perfect}} = 0$ since there is one unique
configuration where all the molecules point in one specific
direction.

Next consider the configuration that differs from the perfect crystal
in that one single molecule have been flipped. That is still pretty
ordered but already in that minor deviation (with $W = N_A$) we get $S
= k_B \ln N_A = \num{7.56d-22}$. The ``next'' state ``$2H$'' yields $W
= N_A(N_A - 1) = \num{1.51d-21}$, already ten times higher. The
expansion of $W_n$ ($n$ being the number of ``heads'') is in fact a
binomial expansion over $N_A$. If we are familiar with the {\em
  binomial theorem}, or willing to acquaint ourselves with it, it is
easy to see the astronomical numbers that will face us going
further.



It is however easy to see that the chances of ever reach other than
fairly random (high entropy) states are disappearingly small, instead,
probability leans toward a situation where will jump back and forth
between the states with the highest likelihood (entropy) as is
predicted by the second law of thermodynamics.



Now, imagine if we start flipping molecules, one at a time, randomly,
as we simulate the back and forth among chemical species in real
life. We will start see entropy increase as we flip molecules. The
only time entropy will decrease is if we happen to flip a ``tails''
back to ``heads''. To end up with another perfect arrangement we would
have to succeed in constantly picking ``tails'' as their numbers grow
smaller and smaller. Every time we would flip a ``heads'' back to
``tails'' entropy would increase again, if only by a miniscule
amount.

Probability theory tells us that we almost certainly never wil
end upp with the (or any of the two) perfect configurations, instead
we're probably changing back and forth among those most probable
configurations, those with the highest entropy. So, we are
spontaneously ending up in states with greater entropy, solely as a
matter of probability, without thermodynamics.















\subsection{Calculating the free energy}


For example, if we want to calculate $\gibbs^0_r$ for the reaction
when hydro peroxide turns to water and oxygen gas,

\ceeqstar{2H2O2(l) -> 2H2O(l) + O2(g)}
we calculate
\ceeqstar{\Delta S^0_r = [2 S^0(H2O) + S^0(O2)] - 2 S^0(H2O)}

We can look up these values in tables and find that

\begin{center}
  \begin{tabular}{ll}
    \toprule
    Species & $S^0$ \\
    & {\footnotesize\si{\joule\per\kelvin\per\mol}} \\
    \midrule
    \ce{H2O} & 70 \\
    \ce{O2} & 205 \\
    \ce{H2O2} & 110 \\
    \bottomrule
  \end{tabular}
\end{center}

We can use these figures to calculate entropy for the reaction as
$\Delta S^0_r = 2 \times 70 + 205 - 2 \times
110~\si{\joule\per\kelvin\per\mol} =
125~\si{\joule\per\kelvin\per\mol}$.

We notice that \dS\ is positive. This can be explained from the fact
that the reactant is a liquid whereas the products are liquid and a
gas and we learned earlier that gasses bring more entropy than do
liquids.

To predict whether this reaction is spontaneous or not we need to
calculate \gibbs for the reaction. We now know that $\Delta S^0_r =
125~\si{\joule\per\kelvin\per\mol}$ and from the lecture we get that
$\enthalpy^0 = -196~\kjpm$ so

\begin{equation*}
  \gibbs^0 = \SI{-196}{\kilo\joule\per\mol}
  - \SI{298.15}{\kelvin}
  \cdot \SI{0.125}{\kilo\joule\per\kelvin\per\mol}
  = \SI{-233}{\kilo\joule\per\mol}
\end{equation*}

Since $\gibbs^0 < 0$ we predict that the reaction occurs
spontaneously.

Note that the amount of the Gibb's free energy does not tell us
anything about the time frame (speed) of a reaction. If $\gibbs = -2$
or $\gibbs = -2000$ does not tell us more than that it is
spontaneous. Thermodynamics does not answer questions about this. The
field of kinetics, however, do, as we will see later.

Now consider melting of ice at room temperature.
\ceeqstar{H2O(s) -> H2O(l)}

We calculate the entropy
\ceeqstar{\dS^0_r = S^0(H2O(l)) - S^0(H2O(l)) = 69.91 - 41.32
  = \SI{28.59}{\joule\per\kelvin\per\mol}}
\dS\ is positive since the reaction is turning a solid into a liquid.

Now we can calculate the free energy
\begin{align*}
  \gibbs_r^0 &= 6.95~\kjpm
  - \SI{298.15}{\kelvin}
  \cdot \SI{2.859d-2}{\kilo\joule\per\kelvin\per\mol} \\
  &= \SI{-1.57}{\kilo\joule\per\mol}\\
\end{align*}

The result tells us that the reaction is spontaneous (at $T =
\SI{198.15}{\kelvin}$). A lot of people know from everyday experience
that this reaction is temperature dependent. We also see that the
reaction is spontaneous even though the enthalpy is positive.




\subsection{Free energy of formation}


We can also calculate the {\em free energy of formation}, $\gibbs_f$,
in a similar way as that of enthalpy of formation.

The standard Gibb's free energy of formation, $\gibbs^0_f$, is defined
by $\gibbs^0_f = \gibbs^0_r$ for formation of one mole of compound
from it's elements in their purest states, under standard conditions
($P = \SI{1}{\bar}, T = \SI{298.15}{\kelvin}$).

These values are tabulated, as was the enthalpy of formation, but the
standard Gibb's free energy of formation can also be calculated
through
\begin{equation*}
  \gibbs^0_f = \enthalpy^0_f - T \dS^0
\end{equation*}


\standardgibbsfreeenergyof is important because it
is a measure of a compunds stability relative to that of it's
elements.

If $\standardgibbsfreeenergyof < 0$ then the compound is
(thermodynamically) stable to it's elements. If we consider how we
started to reason about Gibb's free energy, we learned that a reaction
was going to be spontaneous if $\gibbs_r < 0$. From this we can draw
the conclusion that the elements will spontaneously form the
compound.

Conversely, if $\standardgibbsfreeenergyof > 0$, the compound is
thermodynamically unstable relative to it's elements.



Consider the free energy of formation for benzene.

\ceeq{6C(gr) + 3H2(g) -> C6H6(l)}

The free energy for formation for benzene is
$\standardgibbsfreeenergyof = \SI{124}{\kilo\joule\per\mol}$ so
benzene is unstable relative to it's compounds.

This means that the reverse reaction where we have decomposition of
benzene is going to be stable.

\ceeq{C6H6(l) -> 6C(gr) + 3H2(g)}

med $\standardgibbsfreeenergyof =
\SI{-124}{\kilo\joule\per\mol}$. Note that \standardgibbsfreeenergyof
has the same absolute value but the reverse sign, as it goes in the
other direction. Remember that Gibb's free energy is a state function
so differences between compounds and elements are the same, the
direction is given by the sign od the distance.

Benzene do however exist, so does that mean that the theory is wrong?
No, benzene do decompose but the reaction is very, very slow.

As we said before: Gibb's free energy of a reaction tells us whether
it will happen spontaneously or not. Gibb's free energy tells us
{\em nothing about the rate of that reaction}.






So we have two ways of calculating \gibbs for a reaction.

We can look up and suum up the $\gibbs_f$ for reactants and products,
respectively, and $\gibbs_r$ is then the differences between those
the sums
\begin{equation*}
  \gibbs^0_r = \sum\gibbs^0_f(\text{products}) - \sum\gibbs^0_f(\text{reactants})
\end{equation*}


The other way is to calculate it from
\begin{equation*}
  \gibbs^0_r = \enthalpy^0_r - T\dS
\end{equation*}

This last equation can be usefull when temperature is important. Let
us consider the decomposition of sodium bicarbonate (baking soda).

\ceeq{2NaHCO3(s) -> Na2CO3(s) + CO2(g) + H2O(g)}

We can look up the enthalpy for the reaction and doing so we find that
$\enthalpy_r^0 = \SI{+135.6}{\kilo\joule\per\mol}$ which tells us that
this is an endothermic reaction, it requires heat to occur. We also
can conclude that entropy is increasing ($\dS > 0$) since the left
hand side consists of one solid compound and there are two gasses
among the products, so entropy is increasing through the reaction. The
actual value is indeed positive, $\dS =
\SI{+0.334}{\kilo\joule\per\kelvin\per\mol}$. From this we can
calculate
\begin{equation*}\tag{a}
  \gibbs^0 = \enthalpy^0 - T \dS^0
  = \SI{+135.6}{\kilo\joule\per\mol}
  - T\cdot \SI{+0.334}{\kilo\joule\per\kelvin\per\mol}
\end{equation*}



Assuming that \dS\ and \enthalpy\ are independent of $T$,
$\gibbs^0$ is a linear function of $T$.

At room temperature $\gibbs^0 = \SI{36.1}{\kilo\joule\per\mol}$ so the
reaction will not occur spontaneously under those conditions. From the
equation we can see that
\begin{equation*}
  T > \frac{\SI{+135.6}{\kilo\joule\per\mol}}{\SI{+0.334}{\kilo\joule\per\kelvin\per\mol}} = 
  \SI{406}{\kelvin}
  \implies
  \gibbs^0_r < 0
\end{equation*}

So at temperatures above \SI{406}{\kelvin} (\SI{133}{\celsius}) the
baking soda will start to decompose and release the gasses, for
instance inside some baked goods in an oven.




\begin{hfigure}
  \begin{center}
    \begin{tikzpicture}
      \begin{axis}[
          title={\gibbs when \enthalpy and \dS\ are positive},
          axis lines=middle,
          width=.67\textwidth,
          height=5cm,
          axis lines = middle,
          domain=0:600,
          ymin=-100,
          ymax=200,
          xmin = -50,
          xmax=750,
          %xtick=\empty,
          ytick={0, 100},
          xlabel=$T~(\si{\kelvin})$,
          ylabel=$\gibbs^0 (\kjpm)$,
          clip = false
        ]
        \addplot[blue, samples=100]{135.6 - 0.334*x};
        \draw[red,thin] (406, 10) -- (406, -50) node[at end,below,black]{$T^*$};
        \draw[red,thin] (-10, 135.6) -- (10, 135.6) node[at start,left,black]{$\enthalpy^0$};
      \end{axis}
    \end{tikzpicture}%
  \end{center}
  \caption{The spontaneity of the decomposition of baking soda and
    subsequent release of vapor and carbon dioxide is temperature
    dependent.}
\end{hfigure}


We can notice that whenever the enthalpy and the entropy of a reaction
have the same sign, we can control spontaneity of the reaction by
controlling the temperature.

In that situation we can find a $T^*$ for which $\gibbs^0 = 0$ and
then the reaction will be non-spontaneous ($\gibbs^0 > 0$) for all $T
< T^*$ and the reaction will be spontaneous ($\gibbs^0 < 0$) for all
$T > T^*$. In our example $T^* = \SI{406}{\kelvin}$.

We find $T^*$ by solving $\gibbs^0 = \enthalpy^0 - T \dS^0$ for
$\gibbs^0 = 0$ which give us
\begin{align*}
  \enthalpy^0 - T^* \dS^0 &= 0 \\
  T^* &= \frac{\enthalpy^0}{\dS^0} \\  
\end{align*}
just as we did previously.



When both enthalpy and entropy are negative, we get a reversed
situation.


\begin{hfigure}
  \begin{center}
    \begin{tikzpicture}
      \begin{axis}[
          title={\gibbs when \enthalpy and \dS\ are negative},
          axis lines=middle,
          width=.67\textwidth,
          height=5cm,
          axis lines = middle,
          domain=0:600,
          ymax=150,
          ymin=-200,
          xmin=-50,
          xmax=750,
          %xtick=\empty,
          ytick={0, 100},
          xlabel=$T~(\si{\kelvin})$,
          ylabel=$\gibbs^0 (\kjpm)$,
          clip = false
        ]
        \addplot[blue, samples=100]{-135.6 + 0.334*x};
        \draw[red,thin] (406, 10) -- (406, -50) node[at end,below,black]{$T^*$};
        \draw[red,thin] (-10, -135.6) -- (10, -135.6) node[at start,left,black]{$\enthalpy^0$};
      \end{axis}
    \end{tikzpicture}%
  \end{center}
  
  \caption{The function $\gibbs = \enthalpy - T \dS$ for a negative
    \dS and negative \enthalpy. When \dS is negative, the straight
    line gets an upward slope and, for a negative \enthalpy, becomes
    spontaneous for lower temperatures, but non-spontaneous for
    temperatures above $T^*$.}
\end{hfigure}

Now the reaction is spontaneous for $T < T^*$, as $\gibbs < 0$ and for
$T > T^*$ the products are now less stable relative to their elements,
$\gibbs > 0$, and the reaction is no longer spontaneous.

Finally, if $\enthalpy^0 < 0$ and $\dS^0 > 0$ the spontaneity of the
reaction is independent of temperature. It is spontaneous for all
temperatures since $\gibbs^0 < 0$ for all temperatures. If
$\enthalpy^0 > 0$ and $\dS^0 < 0$, $\gibbs^0 > 0$ for all temperatures
so the reaction is non-spontaneous for all temperatures.


\begin{htable}
  \begin{center}
    \begin{tabularx}{.67\textwidth}{llX}
      \toprule
      Enthalpy & Entropy & Spontaneous or not \\
      \midrule

      $\enthalpy^0 < 0$ & $\dS^0 > 0$ & allways spontaneous,
      independently of $T$, since \gibbs is negative and decreasing. \\

      $\enthalpy^0 > 0$ & $\dS^0 < 0$ & never spontaneous since \gibbs
      is positive and increasing. \\

      $\enthalpy^0 > 0$ & $\dS^0 > 0$ & spontaneous for $T > T^*$ since
      \gibbs is positive and decreasing. \\

      $\enthalpy^0 < 0$ & $\dS^0 < 0$ & spontaneous for $T < T^*$ since
      \gibbs is negative and increasing. \\

      \bottomrule
    \end{tabularx}
  \end{center}
  \caption{ The linear dependence of \gibbs of \enthalpy and \dS
    give us some obvious consequences for \gibbs and spontaneity.  }
\end{htable}



These are simple consequences from the linear dependence between
Gibb's free energy, enthalpy and entropy. If the enthalpy is positive
at \SI{0}{\kelvin} and the entropy is positive the function of the
free energy is going to be a straight line with a downward slope. If
the enthalpy, on the other hand, is negative at \SI{0}{\kelvin} and
the entropy is negative the function of the free energy is going to be
a straight line with a upward slope. In both cases there's a straight
line, crossing the $x$ axis at $T^*$


\addsec{Summary}


A spontaneous reaction proceeds in the forward direction without the
need for outside intervention.

We could reason that if the enthalpy of a reaction is negative that
means that the system around the reaction would reach a more stable
energy level as the reaction proceeds. Enthalpy is however not enough
to predict spontaneity of a reaction as we saw spontaneous reactions
with positive enthalpy as well as non-spontaneous reactions with
negative enthalpy.















\end{document}
