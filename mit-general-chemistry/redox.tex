\documentclass[../mit-general-chemistry.tex]{subfiles}
\begin{document}

\chapter{Reduction-oxidationreactions}


Oxidation/reduction involves equilibrium and it involves
thermodynamics. It is really important for reactions occuring in the
body and acid-base is as well. Between acid-base reactions and
oxidation-reduction they cover alot of how enzymes work.




\section{Oxidation states and balancing reactions}


Guidelines for assigning oxidation states
\begin{enumerate}[label=\arabic*)]
\item In free elements, each atom has an oxidation state of zero.

  Example: \ce{H2}

\item For monoatomic ions the oxidation state is equal to the charge
  of the ion.

  Example: \ce{Li^+} atoms has the oxidation state of $+1$.

\item Group 1 and Group 2 metals have consistent rules. In forming
  compounds, Group 1 metals have an oxidation state of $+1$. Group 2
  metals have an oxidation state of $+2$. Aluminum has an oxidation
  state of $+34$ in all it's compounds.

\item The oxidation state of oxygen in most compounds is
  $-2$. However, in peroxides (e.g. \ce{H2O2} and \ce{O2^2-}) oxygen
  has an oxidation state of $-1$.

\item The oxidation state of hydrogen is $+1$ except when it is bonded
  to metals in binary compounds.

  Example: \ce{LiH}, \ce{NaH} and \ce{CaH2} where \ce{H} has the
  oxidation state of $+1$.

\item Fluorine has an oxidation state of $+1$ in all it's compounds.

  Other halides (compounds containing halogens, \ce{Cl}, \ce{Be}, \ce{I}) have negative oxidation
  states when they occur as halide ions in their compounds
  (e.g. \ce{NaCl} where the chlorine atom has the oxidation state of
  $-1$).

  When halides are combined with oxygen, in oxoacids, they have
  positive oxidation states (e.g. \ce{ClO^-}, hypochlorite).

\item In a neutral molecule, the sum of the oxidation state of all the
  atoms must be zero.

  In a polyatomic ion, the sum of all the oxidation states of all the
  elements in the ion must add upp to the net charge of the ion.  

  \begin{example}
    Consider \ce{NH4^+}.
    
    \paragraphbreak
    
    Oxidation state of \ce{H} is $+1$. Oxidation state of \ce{N} is
    (therefore) $-3$, so that the oxidation states add up to the net
    charge of the ion $+1$.
  \end{example}


\item Oxidation states do not have to be integers. For example, the
  oxidation state of oxygen in \ce{O2^-} is $-\sfrac{1}{2}$.

\end{enumerate}


Those are the rules. Let's look at some examples.


\begin{example}
  Consider \ce{Li2O}. Lithium is a Group 1 metal and has an oxidation
  state of $+1$. Since the molecule is uncharged, oxygen has the
  oxidation state of $-2$.
\end{example}

\begin{example}
  Consider \ce{PCl5}. Chlorine is a halogen (Group 17) and has a
  negative oxidation state. In this compound yje pxodation state of
  $-1$, which make the oxidation state of phophorus $+5$ since the
  molecule is uncharged.
\end{example}

\begin{example}
  Looking at \ce{HNO3}, the oxidtion state of hydrogen is $+1$ in
  compounds. The oxidation state of oxygen is $-2$ except in
  peroxides. This makes the oxidation state of nitrogen $+5$ in this
  compound.
\end{example}





\section{Reductio/oxidation reactions}

\begin{definition}[Oxidation]
  Oxidation is the loss of electrons. Oxidation state goes up.
\end{definition}


\begin{definition}[Reduction]
  Reduction is the gain of electrons. Oxidation state goes down.
\end{definition}


\begin{definition}[Oxidizing agent]
  The oxidizing agent accepts electrons. In a reaction, it is the
  species that oxidizes another species, it is reduced itself.
\end{definition}


\begin{definition}[Reducing agent]
  The reducing agent donates electrons. In a reaction it is the
  species that reduces another species, it is oxidiced itself.
\end{definition}



In a {\em disproportionation reaction}, a reactant in one oxidation
state is both oxidized and reduced.

If we consider the {\em unbalanced} reaction
\ceeq{3NaClO -> NaClO3 + 2NaCl}

If we break the reaction down, the \ce{Na^+} ion is called a {\em
  spectator ion} in that it is not affecting the reaction, it does not
even make it into the two reactions.
\begin{align*}
  &\ce{\underset{\footnotesize +1}{\mathrm{Cl}}O^- -> \underset{\footnotesize +5}{\mathrm{Cl}}O3^-}&(\ce{Cl}~\text{oxidated})\\
  &\ce{ClO^- -> \underset{\footnotesize -1}{\ce{Cl^-}}}&(\ce{Cl}~\text{reduced})\\
\end{align*}

In the product of the left hand side of (a), the oxygen has the
oxidation state od $-2$ which make us assign the chlorine the
oxidation state of $-1$. In the right hand side, chlorine has been
oxidated since it now has the oxidation state of $+5$.

Since chlorine changes oxidation state from $+1$ to $+5$ this make (a)
an oxidation of chlorine. Since chlorine changes oxidation state from
$+1$ to $+1$ to $-1$ in (b), (b) is a reduction of chlorine. We say
that the whole reaction is a disproportionation of chlorine.






\section{Balancing redox reactions in acidic solution}



Now, consider

\ceeq{Fe^2+ + Cr2O7^2- -> Cr^3+ + Fe^3+}

This equation is deliberately unbalanced. We can calculate that the
oxidation state of chromium is reduced from $+6$ to $+3$ and that the
oxidation state of iron is decreased from $+2$ to $+3$ which means
iron is oxidized.

To balance an equation for a reduction/oxidation reaction in an acidic
solution we follow these steps
\begin{enumerate}[label=\arabic*)]
\item Write the two unbalanced half-equations of the species being
  oxidized and reduced
  \begin{align*}
    \ce{Cr2O7^2- &-> Cr^3+}\\
    \ce{Fe^2+ &-> Fe^3+}\\
  \end{align*}

\item Assign coefficients to balance the number of atoms of all
  elements except oxygen and hydrogen on both sides of the equation
  equal.
  \begin{align*}
    \ce{Cr2O7^2- &-> 2Cr^3+}\\
    \ce{Fe^2+ &-> Fe^3+}\\
  \end{align*}

\item Add \ce{H2O} to balance oxygen atoms in the equations
  \begin{align*}
    \ce{Cr2O7^2- &-> 2Cr^3+ + 7H2O}\\
    \ce{Fe^2+ &-> Fe^3+}\\
  \end{align*}

\item Balance hydrogen with \ce{H^+} (some texts use \hydronium ions
  which is more technically correct but more complicated)
  \begin{align*}
    \ce{Cr2O7^2- + 14H^+ &-> 2Cr^3+ + 7H2O}\\
    \ce{Fe^2+ &-> Fe^3+}\\
  \end{align*}

\item Balane the charges on both sides of the equation by adding
  electrons
  \begin{align*}
    \ce{Cr2O7^2- + 14H^+ + 6e^- &-> 2Cr^3+ + 7H2O}\\
    \ce{Fe^2+ &-> Fe^3+ + e^-}\\
  \end{align*}

\item Multiply the half-reactions so that the number of electrons
  donated in the oxidation reaction equals those accepted in the
  reduction reaction
  \begin{align*}
    \ce{Cr2O7^2- + 14H^+ + 6e^- &-> 2Cr^3+ + 7H2O}\\
    \ce{6Fe^2+ &-> 6Fe^3+ + 6e^-}\\
  \end{align*}

\item Add the half-equations together and make the appropriate cancelations
  \begin{align*}
    \ce{6Fe^2+ + Cr2O7^2- + 14H^+ &-> 6Fe^3+ + 2Cr^3+ + 7H2O}\\
  \end{align*}

\item Double check that the left hand side and the right hand side are
  balanced with each other
  \begin{inlinetable}{lrr}
    & left & right \\
    \midrule
    \ce{Fe} & 6 & 6 \\
    \ce{Cr} & 2 & 2 \\
    \ce{O} & 7 & 7 \\
    \ce{H} & 14 & 14 \\
    charge & $+24$ & $+24$ \\
  \end{inlinetable}
\end{enumerate}





\section{Balancing redox reactions in basic solution}

As in the case of acidic solutions, there are different approaches to
balancing equations in basic solution. Here is one.

Consider
\ceeq{Fe^2+ + Cr2O7^2- -> Cr^3+ + Fe^3+}
(again).

\begin{enumerate}[label=\arabic*)]
\item Follow steps 1-7 of the approach for the acidi solution which takes us
  up to this point
  \begin{align*}
    \ce{6Fe^2+ + Cr2O7^2- + 14H^+ &-> 6Fe^3+ + 2Cr^3+ + 7H2O}\\
  \end{align*}

\item ``Adjust the \pH'' by adding \hydroxide to both sides to
  ``neutralize'' the hydronium
  \begin{align*}
    \ce{6Fe^2+ + Cr2O7^2- + 14H^+ + 14\hydroxide &-> 6Fe^3+ + 2Cr^3+ + 7H2O + 14\hydroxide}\\
  \end{align*}
  if we simplify by \ce{H^+ + OH^- -> H2O} and cancel out some water,
  we get
  \begin{align*}
    \ce{6Fe^2+ + Cr2O7^2- + 7H2O &-> 6Fe^3+ + 2Cr^3+ + 14\hydroxide}\\
  \end{align*}
\end{enumerate}





\section{Electrochemical cells}


\begin{definition}
  An {\em electrochemical cell} is a device in which an electrical
  current (a flow of electrons in a circuit) is either produced by a
  spontaneous chemical reaction or is used to bring about a
  non-spontaneous chemical reaction.
\end{definition}

\begin{definition}
  A {\em battery} is, technically, a collection of electrochemical
  cells joined in s series so that the voltage the battery produce is
  the sum of the voltages every cell produce.
\end{definition}



The reaction

\ceeq{Cu^2+(aq) + Zn(s) -> Cu(s) + Zn^2+(aq)}

is a chemical redox reaction. The zinc is being oxidized into zinc
cations and the copper cations are reduced to solid copper. The
oxidation and the reduction of the reaction consists if the two
half-reactions

\begin{align}
  \ce{Zn(s) &-> Zn^2+(aq) + 2e^-} &\text{(oxidation)}\\
  \ce{Cu^2+(aq) + 2e^- &-> Cu(s)} &\text{(reduction)}
\end{align}


During the reaction, electrons are transferred from the reducing agent
to the atoms being oxidized. In a sollution with all the species
present this occurs directly, but we can construct two {\em
  half-cells} where we separate the reactions between two
containers. We take one container with a solution of \ce{ZnSO4} and a
strip of solid zinc immersed into the solution. Then we take another
container with solution of \ce{CuSO4} with a strip of solid copper
immersed into the solution. We connect the metal strips in the
containers with an electrical cord, a metal wire, to facilitate
transfer of electrons between the containers.



One type of electrochemical cell consists of two half-cells in which
one hal-reaction of a reduction-oxidation reaction occurs. Each
half-cell consists of a beaker with an electrode of a metal and a
solution of ions of that metal. The two half-cells are connected by an
electrical cord between the electrodes as well as a salt bridge,

An oxidation reaction occurs in a beaker when a metal atom on the
surface of the electrode gives up an electron to the electrode and
enters the surrounding solution as an cation. A reduction reaction
occurs in a solution when one cation gains an electron as it collides
with the electrode. This converts the cation to a solid metal atom
that attaches to the electrode.

The electrode at which an oxidation occurs is called the anode of the
cell. The electrode at which a chemical reduction reaction occurs is
called the cathode.

The reactions of the cell can be summarized in three equations. For
two metals \ce{X}  and \ce{Y}
\begin{align*}
  \ce{X(s) &-> X^+(aq) + e^-}&\text{(oxidation)}\\
  \ce{Y^+(aq) + e^- &-> Y(s)}&\text{(reduction)}\\
  \ce{X(s) + Y^+(aq) &-> X^+(aq) + Y(s)}\\
\end{align*}


A common example of an electrochemical cell consists of zinc and
copper and a salt bridge of potassium chloride.

An electrode of zinc, the {\em anode}, is immersed in a solution of
sulfuric acid and zinc ions, a copper cathode immersed in another
solution of sulfuric acid and copper ions The cathode and the anode
are connected with an electrical cord. In the illustration below there
is also a volt meter attached to the cord. The two beakers of solution
are also connected by a ``salt bridge'', often consisting of a
gelified electrolyte (\ce{KCl} and \ce{NaCl} are common choices)
encapsulated in a glass tube with open ends. The role of the salt
bridge is to maintain electrical neutrality within the internal
circuit, preventing the cell from rapidly running its reaction to
equilibrium.

\begin{hfigure}
  \begin{center}
    \begin{tikzpicture}[node distance = .1cm]
      \coordinate (coord1) at (0, 0);
      \coordinate[right=4cm of coord1] (coord2);
      \node (anode) at ($(coord1)+(-.5,0)$) [
        draw,
        fill=black!20,
        thick,
        minimum width=.8cm,
        minimum height=2.8cm,
        yshift=.5cm
      ]{\ce{Zn(s)}};
      \node (cathode) at ($(coord2)+(.3,0)$) [
        draw,
        fill=black!20,
        thick,
        minimum width=.8cm,
        minimum height=2.8cm,
        yshift=.5cm
      ]{\ce{Cu(s)}};
      \coordinate (saltbridgeNE) at ($(coord2) + (-.5, -1) + (90:3)$);
      \draw[ultra thick, black!60,line width=2mm] ($(coord1) + (.5, -1)$)
      -- ++(90:3)
      -- (saltbridgeNE)
      node[midway,above]{\footnotesize salt bridge (\ce{KCl})}
      -- ++(270:3);
      \node (flask1) at (coord1) {\includegraphics[width=.25\textwidth]{flask}};
      \node (flask2) at (coord2) {\includegraphics[width=.25\textwidth]{flask}};
      \path (coord1) ++(240:1.4) node(sulfuric1){\ce{SO4^2-}};
      \path (coord1) ++(275:1.2) node(zincion){\ce{Zn^2+}};
      \path (coord2) ++(240:1.4) node(sulfuric2){\ce{SO4^2-}};
      \path (coord2) ++(275:1.2) node(cupperion){\ce{Cu^2+}};
      \draw[ultra thick] (anode.north) -- ++(90:2) coordinate (aboveanode);
      \draw[ultra thick] (cathode.north) -- ++(90:2) coordinate (abovecathode);
      \node[circle,draw,ultra thick] (V) at ($(aboveanode)!0.5!(abovecathode)$) {\Huge V};
      \draw[ultra thick] (V.west) -- (aboveanode)
      node[midway,above] {\ce{e^- ->}};
      \draw[ultra thick] (V.east) -- (abovecathode);
    \end{tikzpicture}
  \end{center}
  \caption{A Daniell cell.}
\end{hfigure}


A cell with copper and zinc electrodes can be described by

\ceeq{Zn^2+(aq) | Zn(s) || Cu(s) | Cu^2+(aq)}

where single bars is notation for phase change and double bars
represents the salt bridge. This particular design of a galvanic cell
with copper and zinc half-cells is called a {\em Daniell cell}.

In this cell the zinc electrode is oxidized into zinc ions, loosing
electrons and the cooper ions are reduced, forming solid cooper on the
cooper electrode.

In each beaker occurs a reduction-oxidation half-reaction
\begin{align*}
  \ce{Cu^2+(aq) + 2e^- &<=> Cu(s)} \\
  \ce{Zn(s) &<=> Zn^2+(aq) + 2e^-} \\
\end{align*}

The electrons in the equations travel through the electrical cord as
apart of the useful work of the cell.

The dynamics and forces of galvanic cells include electron affinity of
the contained metals as well as diffusion and distribution of the
salt ions of the salt bridge. Maybe I will expand on this subject in a
later edition of this text.







\subsubsection{Faraday's laws of electrolysis}

The amounts of copper deposited from the copper ion solution onto the
copper electrode as well as the amount of solid zinc consumed into the
zinc ion solution are proportional to the charge passed through the
electrical cord. This is stated in {\em Faraday's laws of
  electrolysis}, expressed in Equation~\ref{faraday}.

\begin{equation}\label{faraday}
  m = \frac{Q}{F}\frac{M}{z}
\end{equation}
for $m$, the mass of substance (copper or zinc) deliberated/consumed
at the electrodes in grams; $Q$, the total electrical charge in
coulombs; $F = \SI{96485}{\coulomb\per\mol}$, Faraday's constant; $M$,
the molar mass of substance in grams per mole; $z$ is the number of
electrons passed per ion ($z = 2$ for \ce{Zn^2+}, for
instance). $\frac{M}{z}$ is the weight of the substance altered.

{\em Faraday's first law} says that the larger the value of $Q$ the
larger $m$ will be if $M, F$ and $z$ are held constant.

{\em Faraday's second law} says that as $\frac{M}{z}$ grows large, so
will $m$ if $Q, F$ and $z$ are held constant.



\begin{example}
  How much zinc and copper will be redistributed as a current of
  \SI{1.0}{\ampere} flows for \SI{3600}{\second} (one hour)?

  \paragraphbreak

  \begin{enumerate}[label=\arabic*)]

  \item Calculate how much charge has passed during the hour. We use
    \begin{equation*}
      Q = I \cdot t
      = \SI{1.0}{\ampere} \cdot \SI{3600}{\second}
      = \SI{3600}{\coulomb}
    \end{equation*}
    
    \item Find out how many moles of electrons that is needed to
      produce that current.

      We use Faraday's constant, $F = \SI{96485}{\coulomb\per\mol}$,
      to calculate
      \begin{equation*}
        n = \frac{Q}{F} = 
        \frac{\SI{3600}{\coulomb}}{\SI{96485}{\coulomb\per\mol}} =
        \SI{3.66d-2}{\mol}\quad(\SI{3.73d-2}{\mol})
      \end{equation*}

    \item Calculate the number of moles of copper and zinc that have
      been converted in the reactions and convert this number into
      mass (\si{\gram}).

      When a mole of zinc ions are consumed, two moles of electrons
      are passed through the electrical cord since the \ce{Zn^2+} ion
      has a valency of two. The same goes for copper and we calculate
      the number of moles for each metal as
      \begin{equation*}
        n_{\text{metals}} = \frac{n_{\text{electrons}}}{z}
        = \frac{\SI{3.66d-2}{\mol}}{2}
        = \SI{1.83d-2}{\mol}
      \end{equation*}

      Now the atom masses of the metals are

      \begin{inlinetable}{cc}
        & atomic mass (u) \\
        \midrule
        \ce{Zn} & \num{65.38} \\
        \ce{Cu} & \num{63.546} \\
      \end{inlinetable}

      This makes
      \begin{align*}
        &m_{\ce{Zn}}
        = \SI{1.83d-2}{\mol} \cdot \SI{65.38}{\atomicmassunit}
        = \SI{1.19}{\gram} \\
        &m_{\ce{Cu}}
        = \SI{1.83d-2}{\mol} \cdot \SI{63.546}{\atomicmassunit}
        = \SI{1.16}{\gram} \\
      \end{align*}
  \end{enumerate}
\end{example}







\subsubsection{Inert electrodes}



Electrodes are not always consumed or produced during chemical
experiments. We can , for instance, construct a galvanic cell with one
platinum electrode where the reaction of the half-cell is going to be
between \ce{Cr^2+} and \ce{Cr^3+} ions and the platinum electrode will
be inert, it will not react with either ions, nor the solution.

If the cathode is a copper electrode then the reaction of that
half-cell is a reduction with the equation

\ceeq{Cu^2+(aq) + 2e^- -> Cu(s)}

In the anode half-cell there is then, we figure, an oxidation. The
chromium-plus-two ions are oxidized into chromium-plus-three ions
giving up one electron in the process

\ceeq{Cr^2+(aq) -> Cr^3+(aq) + e^-}

This type of cell can be described with the notation

\begin{equation}
  \ce{Pt(s) | Cr^2+(aq), Cr^3+(aq) || Cu^2+(aq) | Cu(s)}
\end{equation}




\subsubsection{Standard hydrogen electrodes}


Hydrogen electrodes are commonly constructed with platinum. Standard
hydrogen electrode (S.H.E.) is a standard against which other
electrode reduction potentials are measured.

The hydrogen electrode is denoted

\begin{align*}
  &\ce{H^+(aq) | H2(g) | Pt(s)} &\text{when it acts as a cathode (\ce{H^+} is reduced)}\\
  &\ce{Pt(s) | H2(g) | H^+(aq)} &\text{when it acts as a anode (\ce{H2} is oxidized)}\\
\end{align*}






\subsection{Cell potentials}

{\em Cell potentials} (\cellpotential) is also called {\em cell
  voltage} as well as {\em electromotive force}. The flow of electrons
arise from a potential difference, \dE, between the electrodes of the
circuit.

The overall free energy of the cell is related to the cell potential
by
\begin{equation}
  \gibbscell = -nF\cellpotential
\end{equation}
for
\begin{itemize}
\item $n$, the number of electrons.
\item $F = \SI{96485}{\coulomb\per\mol}$, Faraday's constant.
\item \cellpotential
\end{itemize}

There is also the corresponding values for electrodes and cells in
their standard states
\begin{equation}\label{standard free energy cell}
  \gibbscellO = -nF\cellpotentialO
\end{equation}
The standard emf, when reactants and products are in their standard
state. The unit of potentials is given in volts (\si{\volt}).



\begin{example}
  Calculate \cellpotentialO for
  \begin{equation*}
    \ce{Zn(s) | Zn^2+(aq) || Cu^2+(aq) | Cu(s)}
  \end{equation*}

  \paragraphbreak

  We calculate the overall \cellpotentialO as the difference between
  the \cellpotentialO of the cathode (as the standard reduction
  potential) and the \cellpotentialO of the anode (as the standard
  reduction potential).

  Standard reduction potentials are looked up in tables, measured
  against S.H.E. The values we find for these particular reactions are
  \begin{inlinetable}{cc}
    Half-reaction & \cellpotentialO (\si{\volt})\\
    \midrule
    \ce{Zn^2+ + 2e^- -> Zn} & \num{-0.7628} \\
    \ce{Cu -> Cu^2+ + 2e^-} & \num{0.3402} \\
  \end{inlinetable}

  and we calculate
  \begin{equation*}
    \cellpotentialO = \num{0.3402} - (\num{-0.7628}) = \SI{1.103}{\volt}
  \end{equation*}
\end{example}


\begin{example}
  Is the flow of electrons between these two reactions going to be
  spontaneous? To answer this we calculate the \gibbscellO of the
  Daniell cell from Equation~\ref{standard free energy cell}.

  \paragraphbreak

  \begin{align*}
    \gibbscellO &= -nF\cellpotentialO = \\
    &= - (2 \cdot \SI{96485}{\coulomb\per\mol} \cdot \SI{1.103}{\volt}) = \\
    &= \SI{-2.128d5}{\kilo\joule\per\mol}
  \end{align*}

  $\gibbscellO < 0$ and as in the chapter of thermodynamics, this
  means that the reaction occurs spontaneously.

  As a general rule of thumb, the flow of electrons of a cell will be
  spontaneous if $\gibbscellO < 0$. Equation~\ref{standard free energy
    cell} tells us that this is the case if $\cellpotentialO > 0$.
\end{example}


A {\em galvanic cell} is an electrochemical cell in which a
spontaneous chemical reaction is used to generate an electrical
current.

In contrast, an {\em electrolytic cell} uses an external electrical
current to drive a non-spontaneous reaction.




\subsubsection{Meaning of the standard reduction potential}


If the standard reduction potential is {\em a large and positive} this
means the element is easy to reduce.


For example, the reaction

\begin{align}
  &\ce{F2(g) + 2e^- -> 2F^-}&E^0 = \SI{2.87}{\volt}
\end{align}
have a great positive standard reduction potential, which makes
fluorine an element that is easy to reduce (it is easy to add
electrons). This makes \ce{F2} a good oxidizing agent. A good
oxidizing agent oxidizes other elements, it gets reduced.

One way to remember this is, for a particular couple, such as
\ce{F2}/\ce{F^-}, with a large, positive standard reduction potential
the {\em oxidized species} (\ce{F2}) is very oxidizing -- is a {\em
  good oxidizing agent}.



A {\em large negative} standard reduction potential means that the
element or compound is hard to reduce.

An example of such a reaction is
\begin{align}
  &\ce{Li^+(aq) + e^- -> Li(s)}&E^0 = \SI{-3.045}{\volt}
\end{align}

The large negative standard reduction potential tell us that it is
hard to add electrons to the lithium ion. Consequently, $E^0\ll 0$
makes $\gibbsO\gg 0$ which means that the reaction is not favoured --
it will thus not occur spontaneously. We say that it is hard to add
electrons to the lithium ion, it is not a good oxidizing agent.

However, \ce{Li(s)} is a good {\em reducing agent} and reduces other
elements or compounds as it gets oxidized itself.

For such a couple, \ce{Li^+}/\ce{Li}, with a large negative standard
reduction potential, {\em the reduced species is very reducing}.


\begin{example}\label{ex:ferric iodine}
  What is the standard reduction potential for
  \ceeqstar{2Fe^3+(aq) + 2I^-(aq) -> 2Fe^2+(aq) + I2(s)}
  \paragraphbreak

  \begin{enumerate}[label=\arabic*)]
  \item We start by looking at what is happening at the cathode. That
    is a reduction. If we look at our equation, we identify the
    reduction as
    \ceeqstar{2Fe^3+(aq) + 2e^- -> 2Fe^2+(aq)}

  \item At the anode the oxidation is happening
    \ceeqstar{2I^- -> I2(s) + 2e^-}

  \item We look up the standard reduction potentials for the two
    half-reactions and find
    \begin{inlinetable}{lc}
      & $E^0$ (\si{\volt}) \\
      \midrule
      reduction & \SI{0.77}{\volt} \\
      oxidation & \SI{0.54}{\volt} \\
    \end{inlinetable}

  \item Now we can calculate the overall cell potential for this
    reaction as
    \begin{equation*}
      \cellpotentialO
      = \Delta E^0_{\text{red}} - \Delta E^0_{\text{oxi}}
      = (0.77) - (0.54) = \SI{0.23}{\volt}
    \end{equation*}

    Since $\cellpotentialO > 0$ we also conclude that the reaction is
    spontaneous under the given circumstances.
  \end{enumerate}
\end{example}


\begin{example}
  Lets consider the reaction of Example~\ref{ex:ferric iodine} again.

  \begin{equation*}
    \ce{2Fe^3+(aq) + 2I^-(aq) -> 2Fe^2+(aq) + I2(s)}
  \end{equation*}
  
  It is an oxidation-reduction reaction in which iron(III) is reduced
  to iron(II) and iodide anion is oxidized into iodine gas.

  \paragraphbreak

  The iron(III) ion is a better oxidation agent than iodine gas and
  the iodide anion is a better reduction agent than the iron(II) ion.

  This is realized from difference between the reduction potentials of
  the half-reactions. The oxidized species of the reaction with the
  bigger positive reduction potential will be the better oxidizing
  agent. Conversely, the better reducing agent of iron(II) ion and
  iodine gas, will be the reduced species from the half-reaction with
  the smallest positive reduction potential.
\end{example}




\section{Chemical and biological oxidation/reduction reactions}


Vitamin \ce{B12} has a large negative reduction potential. How is it
reduced in the body?

This is important as \ce{B12} need to be reduced to be active.

The proper function of an enzyme that requires \ce{B12} and folic acid
is thought to be important in preventing heart disease, birth defects
and the body's mental health maintenance (dementia).

Red meat is a good source of vitamin \ce{B12}. In fact, all meat
contains \ce{B12}. Plants does not contain any \ce{B12} and
vegetarians and vegans should make sure they have enough of it.

Folic acid is found in leafy, green vegetables, orange juice and
Norwegian beer (perhaps Swedish as well).

How is vitamin \ce{B12} reduced in the body? ...






\section{Adding and subtracting half-reactions}



Suppose you nee to use a standard reduction potential for

\ceeq{Cu^2+(aq) + e^- -> Cu^+(aq)\label{copper(II) to copper(I)}}

and it is not given to you. Instead you are given the standard
reduction potential for
\begin{align}
  &\ce{Cu^2+(aq) + 2e^- -> Cu(s)\label{copper(II) to copper}}\\
  &\ce{Cu(s) -> Cu^+(aq) + e^-\label{copper to copper(I)}}
\end{align}

How would you find the standard reduction potentials for
the reaction of Equation~\ref{copper(II) to copper(I)}?

The task is of course set up for us to see that if we ``add'' the last
two equations (\ref{copper(II) to copper} and \ref{copper to
  copper(I)}) together, the result is the reaction of
Equation~\ref{copper(II) to copper(I)}.

If you ``add'' together two reduction-oxidation reactions, the new
standard reduction potential is calculated by using the formula for
Gibb's free energy
\begin{equation*}
  \gibbsO = -nFE^0
\end{equation*}

and we also remember that
\begin{equation*}
  \gibbsO_{\text{new}} = \gibbsO_{\text{reduction}} - \gibbsO_{\text{oxidation}}
\end{equation*}

We can combine these two formulas into
\begin{equation}
  -n_3FE^0_{\text{new}} = -n_1FE^0_{\text{reduction}} + n_2FE^0_{\text{oxidation}}
\end{equation}
to obtain
\begin{equation}\label{eq:E_new}
  E^0_{\text{new}} = \frac{n_1E^0_{\text{reduction}} - n_2E^0_{\text{oxidation}}}{n_3}
\end{equation}
where $n_1$, $n_2$ and $n_3$ are the stoichiometric coefficients
(number of moles) of electrons in the three reactions.

We can use this formula to calculate the standard reaction potential
for a half-reaction that is the sum of two other half-reactions.


\begin{example}
  Calculate the standard reduction potential for the reaction of
  Equation~\ref{copper(II) to copper(I)} given

  \begin{inlinetable}{ll}
    Half-cell reaction & $E^0$ (\si{\volt}) \\
    \midrule
    \ce{Cu^2+(aq) + 2e^- -> Cu(s)} & \num{0.340} \\
    \ce{Cu(s) + e^- -> Cu^+(aq)} & \num{0.522} \\
  \end{inlinetable}

  \paragraphbreak

  We plug in the values of $E^0$ for the reactions into
  Equation~\ref{eq:E_new}
  \begin{equation*}
    E^0_{\text{new}} = 2\cdot\SI{0.340}{\volt} - \SI{0.522}{\volt} = \SI{0.158}{\volt}
  \end{equation*}
\end{example}



When we are using Equation~\ref{eq:E_new} to calculate the cell
potential for a {\em complete cell} (anode and cathode) the number of
electrons of the reduction will be equal to the number of electrons of
the oxidation and these will cancel each other out. and we can
simplify the equation
\begin{equation}
  \Delta E^0_{\text{cell}} = E^0_{\text{cathode}} - E^0_{\text{anode}}
\end{equation}






\section{Nernst equation}


When a battery is exhausted, this is a sign that the
reduction-oxidation of the cell has reached equilibrium. At this point
in time, the cell generates zero potential difference across it's
electrodes. To understand this we need to study how the cell potential
changes with cell composition.


We remind ourselves of what we already know about equilibrium and the
components of reactions.

We know that \gibbs\ changes as the concentrations of reactants and
products change. When equilibrium is reached $\gibbs = 0$.


\end{document}
