\documentclass[../mit-general-chemistry.tex]{subfiles}
\begin{document}

\chapter{Reduction-oxidationreactions}


Oxidation/reduction involves equilibrium and it involves
thermodynamics. It is really important for reactions occuring in the
body and acid-base is as well. Between acid-base reactions and
oxidation-reduction they cover alot of how enzymes work.




\section{Oxidation states and balancing reactions}


Guidelines for assigning oxidation states
\begin{enumerate}[label=\arabic*)]
\item In free elements, each atom has an oxidation state of zero.

  Example: \ce{H2}

\item For monoatomic ions the oxidation state is equal to the charge
  of the ion.

  Example: \ce{Li^+} atoms has the oxidation state of $+1$.

\item Group 1 and Group 2 metals have consistent rules. In forming
  compounds, Group 1 metals have an oxidation state of $+1$. Group 2
  metals have an oxidation state of $+2$. Aluminum has an oxidation
  state of $+34$ in all it's compounds.

\item The oxidation state of oxygen in most compounds is
  $-2$. However, in peroxides (e.g. \ce{H2O2} and \ce{O2^2-}) oxygen
  has an oxidation state of $-1$.

\item The oxidation state of hydrogen is $+1$ except when it is bonded
  to metals in binary compounds.

  Example: \ce{LiH}, \ce{NaH} and \ce{CaH2} where \ce{H} has the
  oxidation state of $+1$.

\item Fluorine has an oxidation state of $+1$ in all it's compounds.

  Other halides (compounds containing halogens, \ce{Cl}, \ce{Be}, \ce{I}) have negative oxidation
  states when they occur as halide ions in their compounds
  (e.g. \ce{NaCl} where the chlorine atom has the oxidation state of
  $-1$).

  When halides are combined with oxygen, in oxoacids, they have
  positive oxidation states (e.g. \ce{ClO^-}, hypochlorite).

\item In a neutral molecule, the sum of the oxidation state of all the
  atoms must be zero.

  In a polyatomic ion, the sum of all the oxidation states of all the
  elements in the ion must add upp to the net charge of the ion.  

  \begin{example}
    Consider \ce{NH4^+}.
    
    \paragraphbreak
    
    Oxidation state of \ce{H} is $+1$. Oxidation state of \ce{N} is
    (therefore) $-3$, so that the oxidation states add up to the net
    charge of the ion $+1$.
  \end{example}


\item Oxidation states do not have to be integers. For example, the
  oxidation state of oxygen in \ce{O2^-} is $-\sfrac{1}{2}$.

\end{enumerate}


Those are the rules. Let's look at some examples.


\begin{example}
  Consider \ce{Li2O}. Lithium is a Group 1 metal and has an oxidation
  state of $+1$. Since the molecule is uncharged, oxygen has the
  oxidation state of $-2$.
\end{example}

\begin{example}
  Consider \ce{PCl5}. Chlorine is a halogen (Group 17) and has a
  negative oxidation state. In this compound yje pxodation state of
  $-1$, which make the oxidation state of phophorus $+5$ since the
  molecule is uncharged.
\end{example}

\begin{example}
  Looking at \ce{HNO3}, the oxidtion state of hydrogen is $+1$ in
  compounds. The oxidation state of oxygen is $-2$ except in
  peroxides. This makes the oxidation state of nitrogen $+5$ in this
  compound.
\end{example}





\section{Reductio/oxidation reactions}

\begin{definition}[Oxidation]
  Oxidation is the loss of electrons. Oxidation state goes up.
\end{definition}


\begin{definition}[Reduction]
  Reduction is the gain of electrons. Oxidation state goes down.
\end{definition}


\begin{definition}[Oxidizing agent]
  The oxidizing agent accepts electrons. In a reaction, it is the
  species that oxidizes another species, it is reduced itself.
\end{definition}


\begin{definition}[Reducing agent]
  The reducing agent donates electrons. In a reaction it is the
  species that reduces another species, it is oxidiced itself.
\end{definition}



In a {\em disproportionation reaction}, a reactant in one oxidation
state is both oxidized and reduced.

If we consider the {\em unbalanced} reaction
\ceeq{3NaClO -> NaClO3 + 2NaCl}

If we break the reaction down, the \ce{Na^+} ion is called a {\em
  spectator ion} in that it is not affecting the reaction, it does not
even make it into the two reactions.
\begin{align*}
  &\ce{\underset{\footnotesize +1}{\mathrm{Cl}}O^- -> \underset{\footnotesize +5}{\mathrm{Cl}}O3^-}&(\ce{Cl}~\text{oxidated})\\
  &\ce{ClO^- -> \underset{\footnotesize -1}{\ce{Cl^-}}}&(\ce{Cl}~\text{reduced})\\
\end{align*}

In the product of the left hand side of (a), the oxygen has the
oxidation state od $-2$ which make us assign the chlorine the
oxidation state of $-1$. In the right hand side, chlorine has been
oxidated since it now has the oxidation state of $+5$.

Since chlorine changes oxidation state from $+1$ to $+5$ this make (a)
an oxidation of chlorine. Since chlorine changes oxidation state from
$+1$ to $+1$ to $-1$ in (b), (b) is a reduction of chlorine. We say
that the whole reaction is a disproportionation of chlorine.






\section{Balancing redox reactions in acidic solution}



Now, consider

\ceeq{Fe^2+ + Cr2O7^2- -> Cr^3+ + Fe^3+}

This equation is deliberately unbalanced. We can calculate that the
oxidation state of chromium is reduced from $+6$ to $+3$ and that the
oxidation state of iron is decreased from $+2$ to $+3$ which means
iron is oxidized.

To balance an equation for a reduction/oxidation reaction in an acidic
solution we follow these steps
\begin{enumerate}[label=\arabic*)]
\item Write the two unbalanced half-equations of the species being
  oxidized and reduced
  \begin{align*}
    \ce{Cr2O7^2- &-> Cr^3+}\\
    \ce{Fe^2+ &-> Fe^3+}\\
  \end{align*}

\item Assign coefficients to balance the number of atoms of all
  elements except oxygen and hydrogen on both sides of the equation
  equal.
  \begin{align*}
    \ce{Cr2O7^2- &-> 2Cr^3+}\\
    \ce{Fe^2+ &-> Fe^3+}\\
  \end{align*}

\item Add \ce{H2O} to balance oxygen atoms in the equations
  \begin{align*}
    \ce{Cr2O7^2- &-> 2Cr^3+ + 7H2O}\\
    \ce{Fe^2+ &-> Fe^3+}\\
  \end{align*}

\item Balance hydrogen with \ce{H^+} (some texts use \hydronium ions
  which is more technically correct but more complicated)
  \begin{align*}
    \ce{Cr2O7^2- + 14H^+ &-> 2Cr^3+ + 7H2O}\\
    \ce{Fe^2+ &-> Fe^3+}\\
  \end{align*}

\item Balane the charges on both sides of the equation by adding
  electrons
  \begin{align*}
    \ce{Cr2O7^2- + 14H^+ + 6e^- &-> 2Cr^3+ + 7H2O}\\
    \ce{Fe^2+ &-> Fe^3+ + e^-}\\
  \end{align*}

\item Multiply the half-reactions so that the number of electrons
  donated in the oxidation reaction equals those accepted in the
  reduction reaction
  \begin{align*}
    \ce{Cr2O7^2- + 14H^+ + 6e^- &-> 2Cr^3+ + 7H2O}\\
    \ce{6Fe^2+ &-> 6Fe^3+ + 6e^-}\\
  \end{align*}

\item Add the half-equations together and make the appropriate cancelations
  \begin{align*}
    \ce{6Fe^2+ + Cr2O7^2- + 14H^+ &-> 6Fe^3+ + 2Cr^3+ + 7H2O}\\
  \end{align*}

\item Double check that the left hand side and the right hand side are
  balanced with each other
  \begin{inlinetable}{lrr}
    & left & right \\
    \midrule
    \ce{Fe} & 6 & 6 \\
    \ce{Cr} & 2 & 2 \\
    \ce{O} & 7 & 7 \\
    \ce{H} & 14 & 14 \\
    charge & $+24$ & $+24$ \\
  \end{inlinetable}
\end{enumerate}





\section{Balancing redox reactions in basic solution}

As in the case of acidic solutions, there are different approaches to
balancing equations in basic solution. Here is one.

Consider
\ceeq{Fe^2+ + Cr2O7^2- -> Cr^3+ + Fe^3+}
(again).

\begin{enumerate}[label=\arabic*)]
\item Follow steps 1-7 of the approach for the acidi solution which takes us
  up to this point
  \begin{align*}
    \ce{6Fe^2+ + Cr2O7^2- + 14H^+ &-> 6Fe^3+ + 2Cr^3+ + 7H2O}\\
  \end{align*}

\item ``Adjust the \pH'' by adding \hydroxide to both sides to
  ``neutralize'' the hydronium
  \begin{align*}
    \ce{6Fe^2+ + Cr2O7^2- + 14H^+ + 14\hydroxide &-> 6Fe^3+ + 2Cr^3+ + 7H2O + 14\hydroxide}\\
  \end{align*}
  if we simplify by \ce{H^+ + OH^- -> H2O} and cancel out some water,
  we get
  \begin{align*}
    \ce{6Fe^2+ + Cr2O7^2- + 7H2O &-> 6Fe^3+ + 2Cr^3+ + 14\hydroxide}\\
  \end{align*}
\end{enumerate}





\section{Electrochemical cells}


\begin{definition}
  An {\em electrochemical cell} is a device in which an electrical
  current (a flow of electrons in a circuit) is either produced by a
  spontaneous chemical reaction or is used to bring about a
  non-spontaneous chemical reaction.
\end{definition}

\begin{definition}
  A {\em battery} is, technically, a collection of electrochemical
  cells joined in s series so that the voltage the battery produce is
  the sum of the voltages every cell produce.
\end{definition}



The reaction

\ceeq{Cu^2+(aq) + Zn(s) -> Cu(s) + Zn^2+(aq)}

is a chemical redox reaction. The zinc is being oxidized into zinc
cations and the copper cations are reduced to solid copper. The
oxidation and the reduction of the reaction consists if the two
half-reactions

\begin{align}
  \ce{Zn(s) &-> Zn^2+(aq) + 2e^-} &\text{(oxidation)}\\
  \ce{Cu^2+(aq) + 2e^- &-> Cu(s)} &\text{(reduction)}
\end{align}


During the reaction, electrons are transferred from the reducing agent
to the atoms being oxidized. In a sollution with all the species
present this occurs directly, but we can construct two {\em
  half-cells} where we separate the reactions between two
containers. We take one container with a solution of \ce{ZnSO4} and a
strip of solid zinc immersed into the solution. Then we take another
container with solution of \ce{CuSO4} with a strip of solid copper
immersed into the solution. We connect the metal strips in the
containers with an electrical cord, a metal wire, to facilitate
transfer of electrons between the containers.



A galvanic cell is one of the simplest types of electrochemical
cell. It consists of two half-cells in which one hal-reaction of a
reduction-oxidation reaction occurs. Each half-cell consists of a
beaker with an electrode of a metal and a solution of ions of that
metal. The two half-cells are connected by an electrical cord
between the electrodes as well as a salt bridge,

An oxidation reaction occurs in a beaker when a metal atom on the
surface of the electrode gives up an electron to the electrode and
enters the surrounding solution as an cation. A reduction reaction
occurs in a solution when one cation gains an electron as it collides
with the electrode. This converts the cation to a solid metal atom
that attaches to the electrode.

The electrode at which an oxidation occurs is called the anode of the
cell. The electrode at which a chemical reduction reaction occurs is
called the cathode.

The reactions of the cell can be summarized in three equations. For
two metals \ce{X}  and \ce{Y}
\begin{align*}
  \ce{X(s) &-> X^+(aq) + e^-}&\text{(oxidation)}\\
  \ce{Y^+(aq) + e^- &-> Y(s)}&\text{(reduction)}\\
  \ce{X(s) + Y^+(aq) &-> X^+(aq) + Y(s)}\\
\end{align*}


A common example of a galvanic cell consists of zinc and copper and a
salt bridge of potassium chloride.

An electrode of zinc, the {\em anode}, is immersed in a solution of
sulfuric acid and zinc ions, a copper cathode immersed in another
solution of sulfuric acid and copper ions The cathode and the anode
are connected with an electrical cord. In the illustration below there
is also a volt meter attached to the cord. The two beakers of solution
are also connected by a ``salt bridge'', often consisting of a
gelified electrolyte (\ce{KCl} and \ce{NaCl} are common choices)
encapsulated in a glass tube with open ends. The role of the salt
bridge is to maintain electrical neutrality within the internal
circuit, preventing the cell from rapidly running its reaction to
equilibrium.

\begin{hfigure}
  \begin{center}
    \begin{tikzpicture}[node distance = .1cm]
      \coordinate (coord1) at (0, 0);
      \coordinate[right=4cm of coord1] (coord2);
      \node (anode) at ($(coord1)+(-.5,0)$) [
        draw,
        fill=black!20,
        thick,
        minimum width=.8cm,
        minimum height=2.8cm,
        yshift=.5cm
      ]{\ce{Zn(s)}};
      \node (cathode) at ($(coord2)+(.3,0)$) [
        draw,
        fill=black!20,
        thick,
        minimum width=.8cm,
        minimum height=2.8cm,
        yshift=.5cm
      ]{\ce{Cu(s)}};
      \coordinate (saltbridgeNE) at ($(coord2) + (-.5, -1) + (90:3)$);
      \draw[ultra thick, black!60,line width=2mm] ($(coord1) + (.5, -1)$)
      -- ++(90:3)
      -- (saltbridgeNE)
      node[midway,above]{\footnotesize salt bridge (\ce{KCl})}
      -- ++(270:3);
      \node (flask1) at (coord1) {\includegraphics[width=.25\textwidth]{flask}};
      \node (flask2) at (coord2) {\includegraphics[width=.25\textwidth]{flask}};
      \path (coord1) ++(240:1.4) node(sulfuric1){\ce{SO4^2-}};
      \path (coord1) ++(275:1.2) node(zincion){\ce{Zn^2+}};
      \path (coord2) ++(240:1.4) node(sulfuric2){\ce{SO4^2-}};
      \path (coord2) ++(275:1.2) node(cupperion){\ce{Cu^2+}};
      \draw[ultra thick] (anode.north) -- ++(90:2) coordinate (aboveanode);
      \draw[ultra thick] (cathode.north) -- ++(90:2) coordinate (abovecathode);
      \node[circle,draw,ultra thick] (V) at ($(aboveanode)!0.5!(abovecathode)$) {\Huge V};
      \draw[ultra thick] (V.west) -- (aboveanode)
      node[midway,above] {\ce{e^- ->}};
      \draw[ultra thick] (V.east) -- (abovecathode);
    \end{tikzpicture}
  \end{center}
  \caption{A galvanic cell.}
\end{hfigure}


A galvanic cell with copper and zinc electrodes can be described by

\ceeq{Zn^2+(aq) | Zn(s) || Cu(s) | Cu^2+(aq)}

where single bars is notation for phase change and double bars
represents the salt bridge.

In this cell the zinc electrode is oxidized into zinc ions, loosing
electrons and the cooper ions are reduced, forming solid cooper on the
cooper electrode.

In each beaker occurs a reduction-oxidation half-reaction
\begin{align*}
  \ce{Cu^2+(aq) + 2e^- &<=> Cu(s)} \\
  \ce{Zn(s) &<=> Zn^2+(aq) + 2e^-} \\
\end{align*}

The electrons in the equations travel through the electrical cord as
apart of the useful work of the cell.




\end{document}
