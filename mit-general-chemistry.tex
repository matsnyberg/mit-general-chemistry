\documentclass[
  fontsize=9.5pt,
  twoside,
  open=right,
  mpinclude=true,
  %parskip=half,
  numbers=noenddot,
  %headings=twolinechapter,
  %chapterprefix=true,
  DIV=9
]{scrbook}

\usepackage[english]{babel}
\usepackage{mats-textbook}
\usepackage{subfiles}
\setlength{\marginparwidth}{1.67\marginparwidth}
\graphicspath{ {mit-general-chemistry/} }

\addbibresource{mit-general-chemistry/mit-general-chemistry.bib}

\begin{document}

% physical constants
%\pgfmathsetmacro\RH{1.097e+7}% per meter
%\pgfmathsetmacro\a0{5.291772106712e−11}%

%\pgfmathdeclarefunction{psi100}{1}{%
%  \pgfmathparse{(2/(\a_0^(3/2))) * e^(-(#1/\a_0)) }%
%}

\title{MIT General Chemistry}
%\date{20/8 2016}
\author{Mats Nyberg}
\maketitle

\frontmatter

\tableofcontents
\clearpage

\mainmatter

\part{Basic properties of matter}
\subfile{mit-general-chemistry/atomic-theory}
\subfile{mit-general-chemistry/periodic-table}
\subfile{mit-general-chemistry/bonding}
\subfile{mit-general-chemistry/shape-of-molecules}

\part{Whether reactions will happen and the direction of reactions}
\subfile{mit-general-chemistry/thermodynamics}
\subfile{mit-general-chemistry/spontaneouschange}
\subfile{mit-general-chemistry/thermodynamics-bio}

\part{Chemical reactions}
\subfile{mit-general-chemistry/chemical-equilibrium}

%\part{Acid-base reaction and redox reactions}
\subfile{mit-general-chemistry/acid-base-equilibrium}
\subfile{mit-general-chemistry/redox}
\subfile{mit-general-chemistry/transition-metals}
\chapter{Kinetics}


\backmatter


\printbibliography
\end{document}




